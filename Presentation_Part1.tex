\documentclass{beamer}
\usepackage[backend=bibtex,style=trad-plain, citestyle=verbose]{biblatex}
\usepackage{xcolor}
% For more themes, color themes and font themes, see:
% http://deic.uab.es/~iblanes/beamer_gallery/index_by_theme.html
%
\mode<presentation>
{
  \usetheme{Darmstadt}       % or try default, Darmstadt, Warsaw, ...
  \usecolortheme{default} % or try albatross, beaver, crane, ...
  \usefonttheme{serif}    % or try default, structurebold, ...
  \setbeamertemplate{navigation symbols}{}
  \setbeamertemplate{}[numbered]
} 
\makeatletter
\setbeamertemplate{footline}
{
  \leavevmode%
  \hbox{%
  \begin{beamercolorbox}[wd=.333333\paperwidth,ht=2.25ex,dp=1ex,center]{author in head/foot}%
    \usebeamerfont{author in head/foot}
  \end{beamercolorbox}%
  \begin{beamercolorbox}[wd=.333333\paperwidth,ht=2.25ex,dp=1ex,center]{title in head/foot}%
    \usebeamerfont{title in head/foot}
  \end{beamercolorbox}%
  \begin{beamercolorbox}[wd=.333333\paperwidth,ht=2.25ex,dp=1ex,right]{date in head/foot}%
    \usebeamerfont{date in head/foot}\insertshortdate{}\hspace*{2em}
    \insertframenumber{} / \inserttotalframenumber\hspace*{2ex} 
  \end{beamercolorbox}}%
  \vskip0pt%
}
\makeatother

\usepackage[english]{babel}
\usepackage[utf8]{inputenc}
\usepackage{chemfig}
\usepackage[version=3]{mhchem}
\usepackage{subfigure}
\usepackage{graphicx}
\usepackage{wasysym}
\usepackage{relsize}
\usepackage{color}

\newcommand{\bigqm}[1][1]{\text{\larger[#1]{\textbf{?}}}}
\newcommand*{\vimage}[2]{\vcenter{\hbox{\includegraphics[#1]{#2}}}}
\newcommand*{\vpointer}{\vcenter{\hbox{\scalebox{2}{\Huge\pointer}}}}

\bibliography{Presentation}
\bibstyle{plain}
\definecolor{mygray}{gray}{0.6}

% On Overleaf, these lines give you sharper preview images.
% You might want to `comment them out before you export, though.
\usepackage{pgfpages}
\pgfpagesuselayout{resize to}[%
  physical paper width=8in, physical paper height=6in]

% Here's where the presentation starts, with the info for the title slide
\title{Experiments}
\author{Daniel Seidinger, Brikena Celaj}
\date{02.11.2018}

\begin{document}

\begin{frame}
  \titlepage
\end{frame}

\begin{frame}{Experiments}
\begin{block}{Zobel 197}
Experiments are an essential part of sound science
\end{block}

\begin{block}{ Zobel 198}
Tests should be fair rather than constructed to support the hypothesis
\end{block}
\end{frame}

\section{Baseline and Data}
\begin{frame}{Baseline}
First steps:
\begin{itemize}
\item Identify an appropriate Baseline
\item Barrier to entry: Must be familiar with existing methods/ideas 
\item BUT: No excuse for poor science!
\end{itemize}
\end{frame}

\begin{frame}{Baseline}
\begin{alertblock}{Possible Problems}
\begin{itemize}
\item Baseline is not updated 
\item Widely available implementation is used as a reference point
$\rightarrow$ Improvements may not be cummulative
\end{itemize}
\end{alertblock}
\end{frame}

\begin{frame}{Persuasive Data}
\begin{itemize}
\item Access and understanding of the appropriate data is critical for experiments
\end{itemize}
\begin{block}{Things to Consider}
\begin{itemize}
\item What data is available, what is the source
\item What specific mechanisms to gather and standardize
\item Will the data be sufficient in volume/quality
\item What domain knowledge is needed to interpret the data
\item Limits, biases, flaws and properties of the data and how this is managed
\item What will the results be if the data supports/opposes the hypothesis 
\end{itemize}
\end{block}
\end{frame}



\begin{frame}{Persuasive Data}
\begin{block}{Do's and Don'ts}

\begin{itemize}
\item Distinguish between observation phase and confirmation phase
\item Choosing parameters to suite data or vice versa invalidates the research
\item Identify data/input for which the hypothesis is least likely to hold, these are the interesting cases
\end{itemize}

\end{block}
\end{frame}

\begin{frame}{Persuasive Data}
\begin{block}{Do's and Don'ts}

\begin{itemize}
\item Before committing to a research question, be sure that you are able to obtain the needed data
\item Have sufficient data sets
\item Test on volumes of data you are making the claims on
\end{itemize}

\end{block}
\end{frame}

\section{Interpretation}
\begin{frame}{Interpretation}
The experiment yielded results (hopefully), what now? \\
\begin{itemize}
\item Are there other possible interpretations?
\item[] $\rightarrow$ Further tests to eliminate them
\item Is the outcome as expected? 
\item Careful with failed experiments
\item  Success in special case $\not =$ Success in general


\end{itemize}
\end{frame}

\begin{frame}{Interpretation}
\begin{itemize}
\item Don't draw undue conclusions or overstate them

\begin{block}{Zobel 204}
[...] many measures in computer
science are on an ordinal, rather than an interval or ratio [...]
\end{block}
\end{itemize}
\end{frame}


\section{Robustness}
\begin{frame}{Robustness}
Properties of robust experiments:
\begin{itemize}

\item Independent of quality of implementation 
\item Yield \color{green}\checkmark \color{black}  or \color{red} $\times$ \color{black} 
\item Demonstrate a trend or pattern
\item Success is obvious, not a subject to interpretation

\end{itemize}
\end{frame}

\begin{frame}{Robustness - Speed Experiments}
\begin{block}{Speed Experiments}
\begin{itemize}
\item Maximum, minimum, average or median time 
\item Care for anomalies 
\item Explain/discuss them
\end{itemize}
\end{block}

\end{frame}

\begin{frame}
\begin{itemize}
\item Compare components to show their value
\item 
\end{itemize}
\end{frame}

\end{document}